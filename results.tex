\section{Computational results}
\label{sec:results}
In this section we present the new implementation for standard basis computation
over Euclidean domains in \singular 4-1-2.\footnote{In the \singular
sources since git commit 7d2091affbf4b4a1a382e5eb0a47f66c0f3c42f7a.}
We compare it to the current implementations in the
computer algebra systems \macaulay (version $1.12$) and \magma (version $2.23$).
For the comparison we use benchmarks with different properties and behaviours.
All examples are computed with respect to the degree reverse lexicographical order.
All algorithms run single threaded, we use an Intel Core i7-6700 CPU with $3.4$ GHz
and $64$GB RAM. The machine runs Arch Linux with unmodified 4.18.12 kernel. For
the examples we refer to~\cite{singular-benchmarks}.

\begin{table}[h]
	\centering
	% \renewcommand{\tabcolsep}{3mm}
  \def\arraystretch{1.2}
  %\scalebox{0.75}{
  % \vspace*{-3mm}
    \begin{tabular}{c||r|r|r|r}
    \toprule
    \multicolumn{1}{c||}{\textbf{Examples}} &
    \multicolumn{1}{c|}{\singular (Thm.~\ref{thm:buchberger-criterion})} &
    \multicolumn{1}{c|}{\singular (all pairs)} &
    \multicolumn{1}{c|}{\macaulay} &
    \multicolumn{1}{c}{\magma}\\
    \midrule
    Cyclic-6 & $0.330$ & $0.320$ & $4.708$ & $2.799$\\
    Cyclic-7 & $18,731.820$ & $5,636.210$ & out of memory & $366.060$\\[0.2em]
    Katsura-7 & $2.200$ & $2.250$ & $204.928$ & $251.630$\\
    Katsura-8 & $133.390$ & $135.360$ & $64,555.420$ & ($>24$h)\\
    Katsura-9 & $13,366.590$ & $12,951.160$ & ($>24$h) & ($>24$h)\\[0.2em]
    Eco-9 & $3.920$ & $4.050$ & $870.409$ & $22.520$\\
    Eco-10 & $38.760$ & $40.670$ & ($>24$h) & $289.540$\\[0.2em]
    F-633 & $0.140$ & $0.120$ &$14.982$ & $12.880$\\
    F-744 & $118.610$ & $117.890$ & ($>24$h) & ($>24$h) \\[0.2em]
    Noon-7 & $34.930$ & $32.700$ & ($>24$h) & ($>24$h)\\
    Noon-8 & $3,1390.060$ & $3,079.370$ & ($>24$h) & ($>24$h)\\[0.2em]
    Reimer-5 & $3.620$ & $3.590$ & out of memory & $1,932.400$\\
    Reimer-6 & $1,216.960$ & $1,232.530$ & out of memory & ($>24$h)\\[0.2em]
    Lichtblau & $1.910$ & $1.830$ & $69.536$ & $2,242.900$\\[0.2em]
    Bayes-148 & $9.970$ & $9.900$ & $117.635$ & $46.240$\\[0.2em]
    Mayr-42 & $212.320$ & $212.770$ & $218.635$ & $40.270$\\[0.2em]
    Yang-1 & $149.120$ & $147.250$ & $181.210$ & $50.330$\\[0.2em]
    Jason-210 & $47.010$ & $46.780$ & ($>24$h) & ($>24$h)\\
    \bottomrule
    \end{tabular}
	\caption{Benchmark timings given in seconds (``($>24$h)'' means that we have stopped the computation
      after at least $24$ hours.)}
	\label{table:gbz-examples}
\end{table}

We can see that \singular's new implementation is always faster than \macaulay.
Comparing it to \magma we see that \magma is way faster for \texttt{Cyclic-7}.
We assume that this is due to \magma using Faug\`ere's F4
algorithm~(\cite{fF41999}). Our implementation considers fewer \spts and \gpts, but
the reduction steps in higher degree are way slower than the linear algebra done
in \magma. We believe that there are more classes of examples where the linear
algebra attempt is more efficient than the polynomial arithmetic used in
\singular, still, we need to investigate this problem in more detail. We can see
that the F4 algorithm seems to be beneficial also for \texttt{Mayr-42} and \texttt{Yang-1}.
Nevertheless, for most of the other examples considered, \singular is by a
factor faster than \magma.

In the above table we list timings for \macaulay's Buchberger implementation
using polynomial arithmetic (as for \singular).
We also tested \macaulay's F4 implementation, but in most of the above examples
it was slower or just as fast as the polynomial arithmetic implementation. Due
to its much higher memory usage, we could not finish most of the examples on the
given machine. The only example where we have seen a better result was Mayr-42
where \macaulay's F4 algorithm finished in $185$ seconds, still using way more
memory than \magma and nearly $10$ times as much as \singular. In the given
example \singular's new algorithm uses, aside from \texttt{Cyclic-7}, the least memory
of all compared implementations.

For \singular we can see that generating all possible \spts and \gpts or only
the ones needed (recall Theorem~\ref{thm:buchberger-criterion}) does not have a
bigger influence on the computation for most of the examples.
Still, for the bigger examples like \texttt{Noon-8} or \texttt{Katsura-9} we see that applying
Theorem~\ref{thm:buchberger-criterion} leads to slightly slower computation. In
\texttt{Cyclic-7} we can even see a huge impact, the computation slows down by a factor
of more than $3$! It sems that having more pairs available at an earlier stage
of the algorithm is advantageous compared to having fewer pairs overall. In most
of the examples the algorithm taking care of all possible pairs does not even
compute more reductions, the product and chain criterion removes those
pairs which are really useless at some later stage. All in all it seems that
considering all possible \spts and \gpts leads to a more stable algorithm.

We found that the application of Lemma~\ref{lem:m2-replace-trick} becomes
sometimes a bottleneck. For example, always exchanging $f$ and $g$ by $\spoly
fg$ and $\gpoly fg$ has a huge drawback in the computation of a basis for
\texttt{Cyclic-7}, leading to a slow down of a factor of more than $3$. We found
that overall it is a good heuristic to apply Lemma~\ref{lem:m2-replace-trick} at
most $5$ times per a single reduction process. The \singular timings in the
above table correspond exactly to this implementation.
