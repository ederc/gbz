\section{How to choose Pairs}
\label{sec:pairs}
Having two classes of polynomials to handle, namely \spts and \gpts we also need
criteria for deciding when such a polynomial is useless in the sense of
predicting a zero reduction or having already a strong standard representation.
The criteria presented in the following are then applied to
Algorithm~\ref{alg:bba} in lines~\ref{alg:bba:update1} and~\ref{alg:bba:update2}
when new elements for the pair set are generated.

A first criterion takes care of useless \gpts:

\begin{lemma}
Let $f,g \in \epr$ such that $\lc f \mid \lc g$ or $\lc g \mid \lc f$. Then $\gpoly f g$ reduces
to zero w.r.t. $\{f,g\}$.
\label{crit:gpoly-easy}
\end{lemma}

\begin{proof}
Since $\gcd\left(\lc f, \lc g\right) = \lc f$ we can choose $b_f = 1$ and $b_g
=0$. Thus $\gpoly f g = 1 \cdot t_f \cdot f - 0 \cdot t_g \cdot g = t_f f$ for monomial
multiples $t_f, t_g$ such that $t_f \hm f = t_g \hm g = \lcm\left(\hm f, \hm
    g\right)$. It follows that $\gpoly f g$ just a multiple of $f$ and
thus reduces to zero w.r.t. $\{f,g\}$.
\end{proof}

As a next step we state well-known criteria by Buchberger, the Product and the
Chain criterion:

\begin{lemma}[Buchberger's Product Criterion]
Let\\
$f,g \in \epr$ be polynomials such that $\hm f$ and $\hm g$ are coprime and $\lc
f$ and $\lc g$ are coprime.
Then $\spoly f g$ reduces to zero w.r.t. $\{f,g\}$
\label{thm:prod-crit}
\end{lemma}

\begin{proof}
The proof is the same as in the field case.
See Theorem~11 in~\cite{lichtblau2012} for more details.
\end{proof}

Note that Lemma~\ref{thm:prod-crit} does \emph{not} apply to \gpts. Take, for
example, $G=\{f,g\}$ with $f=3x$ and $g=2y$. Then $\hm f = x$ and $\hm g = y$
are coprime and $\spoly f g = g f - fg = 0$, but $\gpoly f g = y \cdot 3x - x
\cdot 2y = xy$. Clearly, we cannot reduce $xy$ any further w.r.t. $G$.

\begin{lemma}[Buchberger's Chain Criterion]
Let $G \subset \epr$ finite and let $f,g,h \in G$ be polynomials such that
\begin{enumerate}
\item $\hm f \mid \lcm\left(\hm g, \hm h\right)$ and
\item $\lc f \mid \lcm\left(\lc g, \lc h\right)$.
\end{enumerate}
If $\spoly f g$ and $\spoly f h$ have a strong standard representation w.r.t.
$G$ then also $\spoly g h$ has a strong standard representation w.r.t. $G$.
\label{thm:chain-crit}
\end{lemma}

\begin{proof}
The lemma is proven as in the field case. We want to have an \spt chain
\[\spoly g h = c_{f,g} m_{f,g} \spoly f g + c_{f,h} m_{f,h} \spoly f h\]
for some coefficients $c_{f,g}, c_{f,h}$ and  some monomials $m_{f,g}, m_{f,h}$.
Property $(1)$ ensures proper monomial multiples $m_{f,g}$ and $m_{f,h}$.
Working over the integers we have to take care of the coefficients, too. Thus
property $(2)$ is needed to ensure the existence of proper multiples $c_{f,g}$
and $c_{f,h}$. For more details see, for example, the proof of Theorem~12
in~\cite{lichtblau2012}: There the statement is proven for a strengthened
version of property $(2)$, namely $\lc f \mid \lc g \mid \lc h$ resp. $\lc g
\mid \lc f \mid \lc h$.
\end{proof}

Moreover, we can state a similar criterion for \gpts:

\begin{lemma}
Let $G \subset \epr$ finite and let $f,g,h \in G$ be polynomials such that
\begin{enumerate}
\item $\hm f \mid \lcm\left(\hm g, \hm h\right)$ and
\item $\lc f \mid \gcd\left(\lc g, \lc h\right)$.
\end{enumerate}
Then $\gpoly g h$ has a strong standard representation w.r.t. $G$.
\label{thm:chain-crit-gcd}
\end{lemma}

\begin{proof}
See Theorem~10.11 in~\cite{bwkGroebnerBases1993} and Theorem~13
in~\cite{lichtblau2012}.
\end{proof}

Besides these statements one can ``decide'' for two polynomials $f$ and
$g$ if we need to compute the corresponding \gpt or the corresponding \spt.
This goes back to~\cite{kapur1988} and can be also found as a variant of
Buchberger's criterion as Theorem~10 in~\cite{lichtblau2012}.

\begin{theorem}[Variant of Buchberger's Criterion]
Let $G \subset \epr$ be a finite set of polynomials, let $<$ be a monomial order
on $\epr$. The following are equivalent:
\begin{enumerate}
\item $G$ is a strong standard basis w.r.t. $<$.
\item\label{cond:all} For all $f,g \in G$ $\spoly f g$ and $\gpoly f g$ reduce to zero.
\item\label{cond:only} For all $f,g \in G$ the following hold:
\begin{enumerate}
\item If $\lc f \mid \lc g$ or $\lc g \mid \lc f$ then $\spoly f g$ reduces to zero.
\item If $\lc f \nmid \lc g$ and $\lc g \nmid \lc f$ then $\gpoly f g$ reduces to zero.
\end{enumerate}
\end{enumerate}
\label{thm:buchberger-criterion}
\end{theorem}

Clearly, when implementing the above criteria especially the choice between
Condition~\ref{cond:all} and Condition~\ref{cond:only} in
Theorem~\ref{thm:buchberger-criterion} has a huge influence on the computation:
\begin{enumerate}
\item Depending the choice of the next element in the pair set in
Algorithm~\ref{alg:bba} it is obvious that Condition~\ref{cond:only} places an
emphasis on \gpts. For a pair of polynomials $f,g\in G$ the algorithm tries to
keep of the lead coefficient of the generated polynomial as small as possible.
This process goes on until at some point eventually this smaller lead
coefficient divides $\lc f$ $\lc g$. Then the corresponding \spt is generated
which then removes the whole lead term.
\item If we use Condition~\ref{cond:all} then there might be a lead term
cancellation, i.e. \spt, being handled before the complete reduction process of
the lead coefficient, i.e. handling of \gpts, is finished.
\end{enumerate}

Of course, one can have an influence on the above situation depending on the
choice of the next element from the pair set $P$ in Line~\ref{alg:bba:choose} of
Algorithm~\ref{alg:bba}. Lichtblau notes in~\cite{lichtblau2012} that, until
now,
there is no real comparison between the two attempts due to missing
implementations.

One statement we can make is the following:

\begin{proposition}
Theorem~\ref{thm:buchberger-criterion} already includes
Lemma~\ref{thm:prod-crit}.
\label{prop:prod-crit-useless}
\end{proposition}

\begin{proof}
By Theorem~\ref{thm:buchberger-criterion} we do only consider $\spoly f g$ if either
$\lc f \mid \lc g$ or $\lc g \mid \lc f$. Lemma~\ref{thm:prod-crit} applies only
in the other situations, but there no \spt is generated at all.
\end{proof}

We have implemented Buchberger's algorithm with both variants of Buchberger's
criterion in \singular. In Section~\ref{sec:results} we present more detailed
results.
