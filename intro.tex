\section{Introduction}
\label{sec:intro} In $1964$ Hironaka already investigated
computational approaches towards singularities and introduced the notion of
\stbs for local monomial orders, see, for
example,~\cite{hironaka11964, hironaka21964}.
In~\cite{bGroebner1965, bGroebner1965eng}, Buchberger initiated, in $1965$, the theory
of \gbs for global monomial orders by which many fundamental problems in mathematics, science
and engineering can be solved algorithmically. Specifically, he introduced some key
structural theory, and based on
this theory, proposed the first algorithm for computing \gbs.
Buchberger's algorithm introduced the concept of critical pairs and repeatedly carries out a certain
polynomial operation (called reduction).

Many of those reductions would be
determined as ``useless'' (i.e. no contribution to the output of the algorithm),
but only a posteriori, that is, after an (often expensive) reduction process.
Thus intensive research was carried out, starting with Buchberger, to avoid the
useless reductions via a priori criteria, see,  for
example,~\cite{bGroebnerCriterion1979,buchberger2ndCriterion1985,gmInstallation1988}.

Once the underlying structure is no longer a field, one needs
the notion of strong \gbs resp. strong \stbs. Influential work was done
by~\cite{kapur1988}, introducing the first generalization of Buchberger's
algorithm over Euclidean domains computing strong \gbs. Since then only a few
optimizations has been introduced, see, for example,~\cite{Wienand2011,
    lichtblau2012, eppSigZ2017}.

In this paper we introduce several new optimizations to the computation of
strong \stbs over Euclidean domains. In Section~\ref{sec:notation} we give the
basic notation and introduce the idea of a reduction step, generalized from the
field case. We state
Buchberger's algorithm over Euclidean domains for global and also for local
monomial orders. Section~\ref{sec:pairs} discusses different variants of how to
handle \spts and \gpts, especially generalized variants of Buchberger's product
and chain criterion. Over Euclidean domains like the integers, coefficient swell
and the missing normalization of the lead coefficient play an important role when it
comes to practical and efficient computation. Modular computation are not
possible in general, but we give a new attempt for keeping coefficients at a low
in Section~\ref{sec:coefficients}. In Section~\ref{sec:nf} we finally give an
in-depth discussion on the normal form computation which provides various attempts
like \ltrs and \lcrs. This, again, helps to keep coefficients small and
minimizes the number of polynomials in a basis which have the same
leading monomial by applying efficient gcd computation.
We have done a new implementation of Buchberger's algorithm in the computer
algebra system \singular. In Section~\ref{sec:results} we compare our implementation
with \macaulay and \magma, exploring the impact of the above ideas with
some interesting results.
