\section{Introduction}
\label{sec:intro}
In 1965~\cite{bGroebner1965, bGroebner1965eng}, Buchberger initiated the theory
of \gbs by which many fundamental problems in mathematics, science
and engineering can be solved algorithmically. Specifically he introduced some key
structural theory, and based on
this theory, proposed the first algorithm for computing \gbs.
Buchberger's algorithm introduced the concept of critical pairs and repeatedly carries out a certain
polynomial operation (called reduction).

Many of those reductions would be
determined as ``useless'' (i.e. no contribution to the output of the algorithm),
but only a posteriori, that is, after an (often expensive) reduction process.
Thus intensive research was carried out, starting with Buchberger, to avoid the
useless reductions via a priori criteria, see,  for
example,~\cite{bGroebnerCriterion1979,buchberger2ndCriterion1985,gmInstallation1988}.

About Gr\"obner bases over the integers there is some more recent work by
Lichtblau \cite{lichtblau2012} considering theoretical aspects and applications.
\todo[inline]{write introduction for upcoming stuff, depends on normal form
  section}

% V
% In~\cite{fF52002Corrected} \jcf presented the \Ffive algorithm which uses
% signatures in order to detected redundant computations in advance. For regular
% sequences as input \Ffive does not compute any zero reduction at all. Over the
% years many new variants of \sigb \gb algorithms have been presented, we refer
% to~\cite{eder-faugere-2016} for a survey on this class of algorithms. All these
% algorithms have in common that they compute \gbs in polynomial rings over
% fields.
%
% %Assuming the integers as ground ring,  \gb computations change in the way that
% %one wants
% Over rings, one needs stronger conditions to
% to compute a so-called strong \gbs. Such a basis is achieved by not only
% handling usual critical pairs consisting of \spts, but also so-called \gpts (see for
% example ~\cite{Wienand2011, lichtblau2012}). In 1988, Kandri-Rody and Kapur
% gave first algorithms for computing \gbs over Euclidean
% domains~\cite{kapur1988}.
%
% Here we generalize \sigb \gb algorithms to computation over Euclidean rings, in
% particular, the integers. In Section~\ref{sec:notation} we introduce basics and
% notation. Sections~\ref{sec:criteria} --~\ref{sec:sigdrops} discuss problems and
% possible solutions when computing over Euclidean rings. In particular, we
% present the problem of signature drops and how to keep them at a minimum.
% Moreover, techniques for improving coefficient growth and other computational
% overhead are given.
% In Section~\ref{sec:hybrid} we show how \sigb computation can be
% efficiently used as a prereduction step for a classical \gb computation over
% Euclidean rings and give experimental results in Section~\ref{sec:timings}.
% Section~\ref{sec:finite-rings} covers problems over rings with zero divisors.
