\section{Avoiding Coefficient Swell}
\label{sec:coefficients}
One main problem when computing over the integers is coefficient growth. We
cannot normalize polynomials as usually done over fields. The only method to
keep at least lead coefficients as small as possible from inside the algorithm
is to efficiently compute \gpts as discussed in Section~\ref{sec:pairs}.

The first idea to handle coefficient swell might be to use modular methods as
done over fiels, see, for example,~\cite{arnoldModular2003}. Sadly this concept
is not working over the integers:

\begin{example}
Let $I = \langle 6x, 8x \rangle \subset \Z[x]$. A strong \stb w.r.t. a global
monomial order $<$ is $G = \{2x,6x,8x\}$ where $2x = \gpoly{8x}{6x}$. Next we try to
compute modular standard bases for $\langle 6x, 8x \rangle \subset \F_p[x]$
for some prime number $p > 3$.\footnote{We choose $p>3$ since it should at least
not divide the lead coefficients of the input polynomials.} The corresponding
\stb is $G_p = \{6x,8x\}$: Buchberger's algorithm over $\F_p$ only considers
$\spoly{8x}{6x} = 0$ and terminates afterwards, thus only the initial generators
are added to $G_p$. If
we ensure that Buchberger's algorithm computes a reduced\footnote{A
reduced \stb over a field w.r.t. a global monomial order means that the lead
coefficients of all elements in the basis are $1$ and no lead term divides
any term of any other polynomial in the basis.} \stb we then get $G_p =
\{x\}$. The problem is that in no case we would get $2x \in G_p$ which is
the important element for the strongness of the \stb for $I$ over $\Z$. Even
before applying Hensel lifting or the Chinese remainder theorem the information
needed is lost.

Thus there is no way to compute a strong \stb over $\Z$ via modular computations
over $\F_p$ and lifting techniques in general.
\end{example}

One trick we can do is trying to find monomials or constants in the ideal we
want to compute a strong \stb for. If we can add such elements to the list of
input polynomials of Algorithm~\ref{alg:bba} this can give a huge speed up to
the overall computation. The following lemma gives us a hint on how to do so.

\begin{lemma}
Let $<$ be a monomial order on $\Z[x]$ and let
$I=\langle f_1,\ldots, f_m\rangle \subset \Z[x]$. Let $\tilde I = \langle
f_1,\ldots, f_m\rangle \subset \Q[x]$. If the standard basis $G = \langle 1
\rangle $ for $\tilde I$ w.r.t. $<$ then there exists a constant $c \in \Z$
such that $c \in I$.
\label{lem:syz-coeff}
\end{lemma}

\begin{proof}
Let $G = \langle 1 \rangle \subset \Q[X]$. Consider $J = \langle 1, f_1,\ldots,
f_m\rangle \subset \Q[x]$. Consider the free module $\Q[x]^{m+1}$ with
standard generators $e_0,\ldots,e_m$ together
with the map $\pi: \Q[x]^{m+1} \rightarrow \Q[x]$ defined via $e_0 \mapsto
1$ and $e_i \mapsto f_i$ for all $1 \leq i \leq m$.
Since $1 \in G$ there must exist a syzygy $\sigma \in \Q[x]^{m+1}$ of the
following structure
\[ \sigma = e_0 + \sum_{i=1}^m p_i e_i\]
where $p_i \in \Q[x]$ are polynomials for all $1\leq i \leq m$.
In other words, we can represent $1 = \pi(e_0)$ as a $\Q[x]$-linear
combination of the $f_i = \pi(e_i)$. Moreover, we define
\[c := \lcm\left(\text{all denominators of all
coefficients of all terms of all $p_i$}\right).\]
Thus we get another syzygy $c\sigma$ which corresponds to the equation
\[c = c \cdot 1 = c \cdot \pi(e_0) =\sum_{i=1}^m (c p_i)\cdot \pi(e_i) =
\sum_{i=1}^m (c p_i) \cdot f_i\]
where $c p_i \in \Z[x]$ for all $1 \leq i \leq m$.
By construction it follows that $c \in I$.
\end{proof}

\begin{algorithm}
\caption{RationlPreCheck (\rpc)} 
\label{alg:rpc}
\begin{algorithmic}[1]
\Require{Ideal $I=\langle f_1,\ldots,f_m\rangle \subset \epr$, monomial order
  $<$}
\Ensure{Ideal $J$ such that $J = I$}
\State{$G \gets$ \stb for $\langle f_1,\ldots, f_m\rangle$ in $\Q[x]$ w.r.t. $<$}
\State{$S \gets \syz{\langle 1, f_1,\ldots, f_m\rangle} \subset \sum_{i=0}^m\Q[x]
  e_i$ w.r.t. $<$}
\If{$1 \in G$}
\State{Search for $\sigma \in S$ with $0$th component of the form $c e_0$, $c
  \in \Q$.}
\State{Find multiple $\lambda \in \Z$ such that $\lambda \sigma \in \sum_{i=0}^m
  \Z[x]e_i$}
\State{$J \gets \langle \lambda c, f_1, \ldots, f_m\rangle$}
\EndIf
\State{$\text{\textbf{return }}J$}
\end{algorithmic}
\end{algorithm}

\begin{example}
\label{ex:syz-example}
We give two examples, a small one we do by hand and a bigger one which gives a
real benefit for the overall computational time:
\begin{enumerate}
\item Let $I \subset \Z[x,y]$ be given by $I=\langle x+4, xy+9, x-y+8\rangle$.
We want to compute a strong \stb for $I$ w.r.t. the degree reverse-lexicographical
order $<$. The \stb for $I$ over $\Q$ includes $1$, so we have a constant in the
\stb for $I$ over $\Z$. We compute $\syz{\langle 1, x+4, xy+9, x-y+8\rangle}
\subset \sum_{i=0}^3 \Q[x,y]$ and get the following three syzygies:
\begin{center}
$
\begin{array}{rcl}
\sigma_1 & = & 7 e_0 - (x+4) e_1 + e_2 + x e_3,\\
\sigma_2 & = & (y-4) e_0 - e_1 + x e_3,\\
\sigma_3 & = & (x+4) e_0 - e_1.
\end{array}
$
\end{center}
$\sigma_1$ is the relation from which we can extract the corresponding constant
for $I$:
\begin{center}
$
\begin{array}{rcl}
7 = 7 \pi(e_0) &=& (x+4) \pi(e_1) - \pi(e_2) - x \pi(e_3)\\
  &=& (x+4)(x+4) - (xy+9) - x(x-y+8)\\
  &=& x^2+8x+16 - xy -9 - x^2 + xy -8x.
\end{array}
$
\end{center}
Thus, we add $7$ to the initial generators of $I$ and run
Algorithm~\ref{alg:bba}. We receive a strong \stb $G = \{7, x+4, y-4\} \subset
\Z[x,y]$ for $I$.
\item A bigger example is given as \texttt{rationalPreCheckExample()} in our
publicly available benchmark library~(\cite{singular-benchmarks}): The ideal $I$
we are considering is generated by $70$ polynomials in $\Z[x,y,z]$. We want to
compute a strong \stb for $I$ w.r.t. the degree reverse-lexicographical order $<$.
In Table~\ref{table:syz-example} give some characteristics of the computation
of a \stb for $I$ on an Intel Core
i7-5557U CPU with 16 GB RAM. Note that the computation of
Algorithm~\ref{alg:rpc} over $\Q$ takes $<0.01$ seconds and needs $<0.3$ MB of
memory, so it is
negligible compared to the computational cost over $\Z$. The strong \stb for $I$ computed by
our implementation consists of only $9$ elements
\begin{align*}
G = \Big\{&18, 6z-12, 2y-4, 2z^2+4z+8,.\\
     &yz+z+3, 3x^2z-15x^2, x^2y+3x^2z+x2,\\
     &\left.x^3+10z, x^2z^2-4x^2z-11x^2\right\}.
\end{align*}
From Algorithm~\ref{alg:rpc} we do not directly get the constant $18 \in G$, but
we get a multiple of it: $6,133,248$. Adding this constant to the generators of
$I$ and applying Algorithm~\ref{alg:bba} represents the third column of
Table~\ref{table:syz-example}, whereas a direct application of
Algorithm~\ref{alg:bba} on $I$ is given in column two.
\end{enumerate}
\end{example}

\begin{table}[h]
	\centering
	% \renewcommand{\tabcolsep}{3mm}
  \def\arraystretch{1.2}
  %\scalebox{0.75}{
  % \vspace*{-3mm}
    \begin{tabular}{c||c|c}
    \toprule
    \multicolumn{1}{c||}{\textbf{Characteristics $\setminus$ Algorithms}} &
    \multicolumn{1}{c|}{\sbba} &
    \multicolumn{1}{c}{\rpc + \sbba}\\
    \midrule
    \text{maximal degree} & 13 & 13\\
    \text{\# zero reductions} & 1,130 & 795\\
    \text{\# product / chain criteria} & 1,279 / 2,990 & 826 / 1,925\\
    \text{memory usage (in MB)} & 1.51 & 0.78\\
    \bottomrule
    \end{tabular}
	\caption{Characteristics of \stb computations of Example~\ref{ex:syz-example}}
	\label{table:syz-example}
\end{table}
\begin{remark} \
\begin{enumerate}
\item Clearly, one can generalize Algorithm~\ref{alg:rpc}: If we do not have $1 \in G$
over $\Q$ we might still get a short polynomial, even a monomial whose
corresponding $\Z[x]$ representation can then be recovered from the corresponding
syzygy module. Note that in this case the reconstruction is a bit harder and the
precheck might take longer, since we have a more complex \stb
computation over $\Q$ that does not terminate early.
\item If one has a computer with at least two cores available the usage of
\texttt{parallel.lib} resp. \texttt{task.lib}~(\cite{singular-parallel,singular-tasks})
in \singular might be worthwhile: One could start the
direct computation of \sbba over $\Z$ on one core plus \rpc over $\Q$ on the
other core. Using the \texttt{waitfirst} command one could always ensure that the
fatest possible running time is achieved.
\end{enumerate}
\end{remark}
