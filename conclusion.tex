\section{Conclusion}
We have presented new ideas for computing standard bases over Euclidean domains
without zero divisors.
The implementation of the corresponding algorithms is available in \singular. We
have seen that \singular is in general faster than \macaulay and \magma in
various examples.

Our next steps include an implementation of the new ideas in the open source C library
\gbl which implements Faug\`ere's F4 algorithm~(\cite{gbl}). Doing so we hope to
benefit from the new ideas and the fast linear algebra. Still, not all ideas
presented here are trivial to move to an F4-style algorithm.

Moreover, we still see a lot of zero reductions in higher degree which slow down
the computation. In order to tackle this problem, we work on a more general chain
criterion trying to exploit more of the structure of the input system. Even
more, a further attempt on signature-based computation over Euclidean rings,
see~(\cite{eppSigZ2017}), should be possible.

Finding better heuristics for the application of
Lemma~\ref{lem:m2-replace-trick} depending on the structure of the input systems
is also an interesting topic to study further. If applied in a ``good'' way it
can have a strong impact on the overall computation.

Another topic we are working on is to improve the implementation in \singular
for Euclidean domains with zero divisors. There, special care needs to be
taken of the annihilator polynomials.
