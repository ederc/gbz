\section{Basic notations}
\label{sec:notation}
% We use notation corresponding to~\cite{eder-faugere-2016}:
Let \er be a Euclidean domain without zero divisors.\footnote{The reader can
    feel fre to think of $\er = \Z$.}
% Besides Section~\ref{sec:finite-rings} we assume that \er has no
% zero divisors.
A \emph{polynomial}
in $n$ variables $x_1,\ldots,x_n$ over \er is a finite \er-linear
combination of \emph{terms} $\const_{v_1,\ldots,v_n} \prod_{i=1}^n
x_i^{v_i}$,
$f=\sum_{v}\const_v x^v \defeq
\sum_{v\in\N^n}^{\text{finite}}\const_{v_1,\ldots,v_n} \prod_{i=1}^n
x_i^{v_i}$,
such that $v \in \N^n$ and $\const_v \in \er$.
The \emph{polynomial ring} $\epr \defeq \er[x] \defeq \er[x_1,\ldots,x_n]$
in $n$ variables over $\er$ is the set of all polynomials over $\er$
together with the usual addition and multiplication. For $f=\sum_{v}\const_v
x^v \neq 0 \in\epr$ we define \emph{the degree of $f$} by $\deg(f) :=
\max\left\{v_1+\cdots +v_n \mid \const_v \neq 0\right\}$. For $f=0$ we set
$\deg(f):=
-1$.
% For $m>0$ we denote by $\module$ a \emph{free $\er$-module} and let $\mgen$ be
% the standard basis (unit vectors) in $\module$.
% A \emph{module element} $\alpha\in \module$ can be written as a finite sum
% $\alpha=\sum_{a \in \epr}^{\text{finite}}a\mbasis i$.
% The elements $a\mbasis i$ are the \emph{module terms} of
% $\alpha$. A \emph{module monomial} is an element of $\module$ with exactly
% one term. A module monomial with coefficient 1 is
% \emph{monic}. Neither module monomials nor terms of module elements
% are necessarily monic. Let $\alpha\simeq\beta$ for
% $\alpha,\beta\in \module$ if $\alpha=\const \beta$ for some non-zero $\const
% \in\er$.
%

Let $\ip f \in \epr$ be a finite sequence of polynomials. We define a module
homomorphism $\pi : \module \rightarrow \epr$ by $\mbasis i \mapsto f_i$ for
all $1\leq i \leq \numberGenerators$. We use the shorthand notation $\proj\alpha
\defeq \pi(\alpha) \in \epr$ for $\alpha \in \module$. An element $\alpha \in
\module$ with $\proj\alpha = 0$ is called a \emph{syzygy}. The module of all
syzygies (of $\langle f_1,\ldots, f_m\rangle$) is denoted $\syz{\langle
  f_1,\ldots, f_m\rangle}$.

In the following we discuss computation with respect to different monomial orders:
\begin{definition}
\label{def:orders}
Let $<$ denote a monomial order on \epr. 
\begin{enumerate}
\item $<$ is called \emph{global} if $x^\alpha \geq 1$ for all $\alpha \in \N^n$.
\item $<$ is called \emph{local} if $x^\alpha \leq 1$ for all $\alpha \in \N^n$.
\item Moreover, we call $<$ \emph{mixed} if there exist $\alpha, \beta \in \N^n$ such
that $x^\alpha \leq 1 \leq x^\beta$.
\end{enumerate}
\end{definition}
% \begin{convention}
% In this paper we restrict ourselves to the \emph{po\-si\-tion-over-term order} on
% $\module$: Let $<$ be a monomial order on $\epr$ and let $a e_i, b e_j$ be two module
% monomials in $\module$. Then we define $\potl$ via
% $a e_i \potl b e_j \text{ if and only if either }i < j \text{ or }i = j \text{
%   and }a < b.$
% Moreover, if not otherwise noted, we assume $<$ to denote the \emph{degree reverse
%   lexicographical order} on \epr.
% \end{convention}
Given such a monomial order $<$ we can highlight the maximal terms of
elements in \epr w.r.t. $<$: For $f\in \epr\ \backslash\ \{0\}$,
$\hd f$ is the \emph{lead term}, $\hm f$ the \emph{lead monomial}, and $\lc f$
the \emph{lead coefficient} of $f$. For any set $F \subset \epr$ we define
the \emph{lead ideal} $L(F) = \langle \hd f \mid f \in F\rangle$; for an ideal
$I \subset \epr$,\  $L(I)$ is defined as the ideal of lead terms of all elements of
$I$. Moreover, we define the \emph{ecart} of $f$ by $\ecart f := \deg(f) -
\deg\left(\hm f\right)$. 
% For $\alpha \in \module\ \backslash\ \{0\}$ we
% denote the maximal term w.r.t. $<$ by $\sig\alpha$, the \emph{signature of
% $\alpha$}. We also define signatures of polynomials: For $f\in \epr$, $\sig f$
% is \emph{a} signature of $f$, given by the signature of the corresponding module
% element $\alpha\in\module$ mapping to $f$: $\sig f = \sig{\proj\alpha} =
% \sig\alpha$. Note that whereas $\sig\alpha$ is uniquely defined, $\sig f$ is not
% since there exist different module elements with different signatures that map
% to $f$.

Working over a field there are many equivalent definitions of how to obtain a
canonical or normal form when reducing a given polynomial by a Gr\"obner basis $G$.
Working over more general rings these definitions are no longer equivalent and
over Euclidean domains, like the integers, this, in particular, results in the term
of \emph{strong}ness we give a meaning in the following:

Assuming that our coefficient ring \er is an Euclidean domain we can define a total
order $\prec$ using the Euclidean norm $|\cdot|$ of its elements: Let $\const_1, \const_2
\in \er$, then $\const_1 \prec \const_2 \text{ if } |\const_1 | < | \const_2 |.$
%\mid \text{ (ties broken corespondingly)}.\]
For example, for the integers we can use the absolute value and break ties via
sign:
\[0\prec -1 \prec 1 \prec -2 \prec 2 \prec -3 \prec 3 \prec \ldots \]

The reduction process of two polynomials $f$ and $g$ in \epr depends now on the
uniqueness of the minimal remainder in the division algorithm in \er:

\begin{definition}
\label{def:reduction}
Let $f, g \in \epr$ and let $G= \{g_1,\ldots,g_r\} \subset \epr$ be a finite set of
polynomials.
\begin{enumerate}
\item We say that \emph{$g$ top-reduces $f$} if $\hm g \mid \hm f$ and if there exist
$a,b \in \er$ such that $\lc f = a\, \lc g +b$ such
that $a \neq 0$, which coincides with $b \prec \lc f$. The top-reduction of $f$
by $g$ is then given by
\[f - a \frac{\hm f}{\hm g} g.\]
So a top-reduction takes place if the reduced polynomial will have either a smaller lead
monomial or a smaller lead coefficient.
\item Relaxing the reduction of the lead term to any term of $f$, we say
that \emph{$g$ reduces $f$}. In general, we speak of the reduction of a
polynomial $f$ w.r.t. a finite set $F\subset \epr$.
Let 
\item We say that $f$ has a \emph{weak standard representation} w.r.t. $G$ if
$f = \sum_{i=1}^r h_i g_i$ for some $h_i \in \epr$ such that $\hm f = \hm{h_j
  g_j}$ for some $j \in \{1,\ldots,r\}$.
\item We say that $f$ has a \emph{strong standard representation} w.r.t. $G$ if
$f = \sum_{i=1}^r h_i g_i$ for some $h_i \in \epr$ such that $\hm f = \hm{h_j
  g_j}$ for some $j \in \{1,\ldots,r\}$ and $\hm f > \hm{h_k g_k}$ for all $k
  \neq j$.
\end{enumerate}
\end{definition}

This kind of reduction is equivalent to CP3 from~\cite{kapur-cp3} and generalizes
Buchberger's attempt from~\cite{buchberger2ndCriterion1985}.

The result of such a reduction might not be unique. This uniqueness is exactly
the property \emph{\stbs} give us. Before defining \stbs, let us give a short note on
the naming convention in this paper:

\begin{convention}
Note that the term \emph{\gb} was introduced by Buchberger in 1965 for bases
w.r.t. a global monomial order (\cite{bGroebner1965, bGroebner1965eng}). Independently, Hironaka
(\cite{hironaka11964, hironaka21964}), and, again
independently, Grauert (\cite{grauert1972}), developed an analgous
concept, called \emph{\stb}, for multivariate power series, i.e. for polynomial
rings equipped with a local monomial order. For this paper we decided to use
the notion \emph{\stb} since it is nowadays the more general one.
\end{convention}

\begin{definition}
\label{def:strong-sb}
A finite set $G \subset \epr$ is called a \emph{\stb} for an ideal $I$ w.r.t.
a monomial order $<$ if $G \subset I$ and $L(G) = L(I)$. Furthermore, $G$ is called a
\emph{strong \stb}
% \footnote{Note that in the area of \sigb algorithms the notion
% of a \emph{strong \gb} is sometimes also used in a different manner, see, for example,
%    \cite{gvwGVW2015,eder-faugere-2016}.}
   if for any $f \in I\backslash\{0\}$ there exists a $g\in G$
such that $\hd g \mid \hd f$. 
\end{definition}

\begin{remark}
Note that $G$ being a strong \stb is equivalent to all elements $g \in G$ having
a strong standard representation w.r.t. $G$. See, for
example, Theorem~1 in~\cite{lichtblau2012} for a proof.
\end{remark}

Clearly, assuming the field case, any \stb is a strong \stb. But in our setting
with \er being an Euclidean ring one has to check the coefficients, too, as
explained in Definition~\ref{def:reduction}.

\begin{example}
\label{ex:stronggb}
Let $\er = \Z$ and $I= \langle x \rangle \in \er[x]$. Clearly, $G :=
\{2x,3x\}$ is a \stb for $I$: $L(I) = \langle x \rangle$ and $x = 3x-2x \in
L(G)$. But $G$ is not a strong \stb for $I$ since $2x \nmid
x$ and $3x \nmid x$.
\end{example}

In order to compute strong \stbs we need to consider two different types of
special polynomials:

\begin{definition}
\label{def:spoly}
Let $f,g \in \epr$. We assume w.l.o.g. that $\lc f \prec \lc g$.
Let $t = \lcm\left(\hm f, \hm g\right)$, $t_f = \frac{t}{\hm f}$, and $t_g = \frac{t}{\hm
  g}$.
\begin{enumerate}
\item Let $a = \lcm\left(\lc f, \lc g\right)$, $a_f = \frac{a}{\lc g},$ and $a_g =
\frac{a}{\lc f}$. The \emph{\spt} of $f$ and $g$ is denoted 
\[\spoly f g = a_f t_f f - a_g t_g g.\]
\item Let $b = \gcd\left(\lc f, \lc g\right)$ . Choose $b_f, b_g \in \er$ such that $b = b_f \lc f + b_g \lc
g$.\footnote{Since $\er = \Z$ is an Euclidean ring the extended gcd always exists.}
The \emph{\gpt} of $f$ and $g$ is denoted 
\[\gpoly f g = b_f t_f f + b_g t_g g.\]
\end{enumerate}
\end{definition}

\begin{remark} \
\label{rem:gpairs}
\begin{enumerate}
\item In the field case we do not need to consider \gpts at all since we can
always normalize the polynomials, i.e. ensure that $\lc f = 1$.
\item Note that $\gpoly f g$ is not uniquely defined: Working over $\er = \Z$ we
know that we can write $\langle \lc f, \lc g \rangle$ as a principal ideal, say
$\langle c \rangle = \langle \lc f, \lc g \rangle$ for some $c\in\er$. Then
there exist $c_f \neq c_f'$, $c_g \neq c_g' \in \er$ such that
\[c_f \lc f + c_g \lc g = c = c_f' \lc f + c_g' \lc g.\]
Depending on the implementation of the $\gcd$ algorithm one specific choice is
made for each \gpt.
% \item Furthermore, if $\lc f \mid \lc g$ then $b_f
% = 1 $ and $b_g = 0$, thus $\gpoly f g$ is just a power product multiple of $f$.
% Thus, in a \stb computation it reduces to zero and is redundant.
\end{enumerate}
\end{remark}

From Example~\ref{ex:stronggb} it is clear that the usual Buchberger
algorithm as in the field case will not compute a strong \stb as we would only
consider $\spoly{2x}{3x} = 3 \cdot 2x - 2 \cdot 3x = 0$. Luckily, we
can fix this via taking care of the corresponding \gpt:
\[\gpoly{2x}{3x} = (-1) \cdot 2x - (-1) \cdot 3x = x.\]
It follows that given an ideal $I \subset \epr$ a strong \stb for $I$ can now be computed using
Buchberger's algorithm taking care not only of \spts but also of \gpts. We refer,
for example,  to~\cite{lichtblau2012} for more details.

In Algorithm~\ref{alg:bba} we give pseudo code for a generic Buchberger
algorithm over the integers. Here, no criterion for detecting useless elements
is applied. This is the topic of the next section. But what is necessary to
discuss beforehand is how to get strong standard representations of elements
handled by Algorithm~\ref{alg:bba}. This is the concept of a \emph{normal
form}:

\begin{definition}
Let $<$ be a monomial order on $\epr$.
Let $\mathcal G$ denote the set of all finite subsets $G \subset \epr$. We call
the map
\begin{center}
$
\begin{array}{rccc}
\text{\sffamily{NF}}: &\epr \times \mathcal G &\longrightarrow& \epr\\
                     & (f, G) &\longmapsto& \nf f G,
\end{array}
$
\end{center}
a \emph{weak normal form} w.r.t. $<$ if for all $f\in \epr$ and all $G \in
\mathcal G$ the following hold:
\begin{enumerate}
\item $\nf 0 G = 0$.
\item If $\nf f G \neq 0$ then $\hd{\nf f G} \notin L(G)$.
\item If $f \neq 0$ then there exists a unit $u\in \epr$ such that either
$uf = \nf f G$ or
$r = uf - \nf f G$ has a strong standard representation w.r.t. $G$.
\end{enumerate}
A weak normal form {\sffamily NF} is called a \emph{normal form} if we can always
choose $u=1$.
\end{definition}

Next we give algorithms that compute normal forms. For their correctness we
refer to Section~$1.6$ in~\cite{gpSingularBook2007}. Algorithm~\ref{alg:nfg}
presents a normal form algorithm for computation w.r.t. a global monomial order
$<$:

\begin{algorithm}
\caption{Normal form w.r.t. a global monomial order $<$
  (\nfn)} 
\label{alg:nfg}
\begin{algorithmic}[1]
\Require{Polynomial $f \in \epr$, finite subset $G\subset \epr$}
\Ensure{\nfn of $f$ w.r.t. $G$ and $<$}
\State{$h \gets f$}
\While{$\left(h \neq 0 \text{ and }G_h := \{g \in G \mid g \text{ top-reduces } h\}
  \neq\emptyset\right)$}
\State{Choose $g \in G_h$.}
\State{$h\gets $ Top-reduction of $h$ by $g$ (see
    Definition~\ref{def:reduction})}
\EndWhile
\State{$\text{\textbf{return }}h$}
\end{algorithmic}
\end{algorithm}

Note that Algorithm~\ref{alg:nfg} may enter an infinite \texttt{while} loop if applied
to local monomial orders. Let us illustrate this behaviour with the ``standard''
example. 

\begin{example}
Let $\epr = K[x]$ where $K$ is a field. We equip $\epr$ with a local monomial
order $<$, i.e. $x < 1$. We set $G = \{g\}$ where $g=x-x^2$ and we want to
compute $\nf f G$ where $f = x$. Using Algorithm~\ref{alg:nfg} we start setting
$h := f$ and find that $g$ top-reduces $h$:
\[h := x - \left(x-x^2\right) = x^2.\]
Now we can again top-reduce $h$ via subtracting $xg$:
\[h := x^2 - x\left(x-x^2\right) = x^3.\]
This process does not stop, but constructs a power series equation:
$x - \left(\sum_{i=0}^\infty x^i\right) \left(x-x^2\right) = 0.$
Since $x<1$ we know that $\sum_{i=0}^\infty x^i = \frac{1}{1-x}$. We see that
Algorithm~\ref{alg:nfg} computes correctly since $(1-x) x = x-x^2$,
still, it is not able to find the finite expression of the power series.
\end{example}

In~\cite{mora82} Mora gave the first attempt to achieve a terminating normal
form algorithm also for local monomial orders:

\begin{algorithm}[h]
\caption{Mora's normal form algorithm w.r.t. a local monomial order $<$
  (\nfn)} 
\label{alg:nfl}
\begin{algorithmic}[1]
\Require{Polynomial $f \in \epr$, finite subset $G\subset \epr$}
\Ensure{\nfn of $f$ w.r.t. $G$ and $<$}
\State{$h \gets f$}
\State{$T \gets G$}
\While{$\left(h \neq 0 \text{ and }T_h := \{g \in G \mid g \text{ top-reduces } h\}
  \neq\emptyset\right)$}
\State{Choose $g \in T_h$ with $\ecart g$ minimal.}
\If{$\left(\ecart g > \ecart h\right)$}
\State{$T \gets T \cup \{h\}$}
\EndIf
\State{$h\gets $ Top-reduction of $h$ by $g$ (see
    Definition~\ref{def:reduction})}
\EndWhile
\State{$\text{\textbf{return }}h$}
\end{algorithmic}
\end{algorithm}

\begin{example}
\label{ex:local-order}
Let $\epr = \Z[x,y]$. A strong \stb for the ideal $I=\left\langle 6+y+x^2,
4+x\right\rangle \subset \epr$ w.r.t. negative degree reverse
lexicographical order $<$ is given by
\[G = \left\{2-x+y+x^2, x-2y-x^2-xy-x^3\right\}.\]
Since $4+x \in I$ we assume that $\nf {4+x} G = 0$. In
Table~\ref{table:infinite-nf} we state Mora's normal form computation
with notation as in Algorithm~\ref{alg:nfl}, i.e. if we have a choice for the
reducer, we take the one of minimal possible ecart.
We start with $h=4+x$ and $T_h =
G$.
\begin{table}[h!]
	\centering
	% \renewcommand{\tabcolsep}{3mm}
  \def\arraystretch{1.2}
  %\scalebox{0.75}{
  % \vspace*{-3mm}
    \begin{tabular}{c|c|c}
    \toprule
    \multicolumn{1}{c|}{$h$} &
    \multicolumn{1}{c|}{$g$} &
    \multicolumn{1}{c}{$h$ \emph{added to} $T_h$?}\\
    \midrule
    $4+x$ & $2-x+y+x^2$ & \checkmark\\
    $3x-2y-2x^2$ & $4+x$ & - \\
    $-x -2y - 3x^2$ & $x-2y-x^2-xy-x^3$ & \checkmark\\
    $-4y - 4x^2 - xy - x^3$ & $4+x$ & -\\
    $- 4x^2 - x^3$ & $4+x$ & -\\
    $0$ & - & -\\
    \bottomrule
    \end{tabular}
	\caption{Local normal form computation due to Mora}
	\label{table:infinite-nf}
\end{table}

Note the importance of $4+x$ being added to $T_h$. Without $4+x$ as reducer the
reduction process would not terminate.
\end{example}

\begin{remark}
Sometimes it can be more efficient to not only add the current status of $h$ to
$T_h$ as a new reducer, but also to generate \gpts with $h$. Doing so,
several lead coefficient reductions may be done in one step. See
Section~\ref{sec:nf} for more details.
\end{remark}
Now we are ready to state Buchberger's algorithm. For the theoretical background
of the algorithm (Buchberger's criterion) we refer to
Theorem~\ref{thm:buchberger-criterion} in Section~\ref{sec:pairs}.

\begin{algorithm}
\caption{Buchberger's algorithm for computing strong \stbs
  (\sbba)} 
\label{alg:bba}
\begin{algorithmic}[1]
\Require{Ideal $I=\langle f_1,\ldots,f_m\rangle \subset \epr$, monomial order
  $<$, normal form algorithm \nfn (depending on $<$)}
\Ensure{\gb $G$ for $I$ w.r.t. $<$}
\State{$G \gets \{f_1,\ldots,f_m\}$}
\State{$P \gets \left\{\spoly{f_i}{f_j}, \gpoly{f_i}{f_j} \mid 1 \leq i < j \leq
  m\right\}$}\label{alg:bba:update1}
\While{$\left(P \neq \emptyset\right)$}
\State{Choose $h \in P$, $P \gets P \setminus \{h\}$}\label{alg:bba:choose}
\State {$h \gets \nf h G$}
\If{$\left(h \neq 0\right)$}
\State{$P \gets P \cup \left\{\spoly{g}{h}, \gpoly{g}{h} \mid g \in
  G\right\}$}\label{alg:bba:update2}
\State{$G \gets G \cup \{h\}$}
\EndIf
\EndWhile
\State{$\text{\textbf{return }}G$}
\end{algorithmic}
\end{algorithm}
