\section{Normal Form Computations}
\label{sec:nf}
Let us recall Definition~\ref{def:reduction} and give the two different
types of a reduction step a name:

\begin{definition}
Let $f \in \epr$ and let $G \subset \epr$ be a finite subset.
\begin{enumerate}
\item If there exists $g\in G$ such that $\hm g \mid \hm f$ and $\lc g \mid \lc
f$ then $f - \frac{\lc f}{\lc g} \frac{\hm f}{\hm g} g$ is a \emph{top-\ltr} of
$f$ (w.r.t. $g$).
\item If there exists $g\in G$ such that $\hm g \mid \hm f$, $\lc g \prec \lc f$ and $\lc g \nmid \lc
f$ then $f - a \frac{\hm f}{\hm g} g$ with $\lc f = a\, \lc g + b$ where $a,b
\in \Z$, $b \neq 0$ amd $b \prec \lc f$ is a \emph{top-\lcr} of $f$ (w.r.t. $g$).
\end{enumerate}
\end{definition}

First we can note that it is enough to consider \ltrs since \lcrs are taken care
of when generating new pairs:

\begin{lemma}
\label{lem:ltrs-are-enough}
Algorithm~\ref{alg:bba} terminates and computes a correct strong \stb for a
given set of geneators and a given momonmial order if we change the
corresponding normal form algorithms to consider only \ltrs.
\end{lemma}

\begin{proof}
We need to show that \lcrs are considered when adding new \gpts to the pair set
$P$. Assume $h$ is the outcome of an \ltr. If there exist possible \lcrs for $h$
w.r.t. $G$ then there is a $g \in G$ such that $\hm g \mid \hm h$, $\lc g
\nmid \lc h$ and $\lc h \nmid \lc g$. Thus $\gpoly h g$ is generated:
\[\gpoly h g = a h + b t g\]
where $a, b \in \Z$ such that $\gcd\left(\lc h, \lc g\right) = a \lc f + b \lc
g$ and $t = \frac{\hm h}{\hm g}$. Now we distinguish two cases:
\begin{enumerate}
\item\label{first-case} If $a=1$ then $\gpoly h g = h - b t g$ which is exactly the corresponding
\lcr of $h$ w.r.t. $g$.
\item If $a \neq 1$ then the \lcr of $h$ w.r.t. $g$, $h' = h - ctg$ for some
$c\in \Z, t\in\epr$, coresponds to the first step
of the Euclidean algorithm calculating $\gcd\left(\lc h, \lc g\right)$. If there
is no further reduction of $h'$ then Algorithm~\ref{alg:bba} generates a corresponding
\gpt between $h'$ and $g$. It follows that
\[h - ctg + \gpoly {h'} g = \gpoly h g.\]
If $h'$ is further reducible by some $g' \in G$ then we note the following:
First of all $g' \in G\setminus \{g\}$ since $\lc g \succ \lc{h'}$ thus there
cannot exist a \ltr or \lcr of $h'$ w.r.t. $g$. Now the reduction of $h'$ by
$g'$ is given as
\[h' - c't'g' = h - ctg - c't'g'\]
for some $c' \in \Z$ and $t' \in \epr$. Since $\hm h = \hm{tg} = \hm{t'g'}$ we
conclude that $ctg + c't'g'$ corresponds to a multiple of either $\spoly
{g}{g'}$ or $\gpoly {g}{g'}$ depending on divisibility of $\lc g$ and $\lc{g'}$.
Nonetheless, once we have a standard representation for $\spoly{g}{g'}$ or
$\gpoly{g}{g'}$ we can reset $h$ by $h - ctg - c't'g'$. Either the lead term or
the lead coefficient of $h$ increases in this process. Thus after finitely many
steps this process terminates and we reach either case~(\ref{first-case}) or
there is no further \lcr reduction possible.
\end{enumerate}
\end{proof}

This was the usual way \singular implemented standard basis reduction over the
integers. Clearly, this is due to historical reasons where the implementation
over $\Z$ was only thought of as a slight generalization of the computation
over fields.

\begin{example}
Recall Example~\ref{ex:local-order} from Section~\ref{sec:notation}: If we
allow only \ltrs when reducing $h = 4+x$ w.r.t. the strong \stb $G = \left\{2-x+y+x^2,
x-2y-x^2-xy-x^3\right\}$ then we get the following reduction table with $T_h =
G$ at the beginning.\begin{table}[h]
  \centering
  % \renewcommand{\tabcolsep}{3mm}
  \def\arraystretch{1.2}
  %\scalebox{0.75}{
  % \vspace*{-3mm}
    \begin{tabular}{c|c|c}
    \toprule
    \multicolumn{1}{c|}{$h$} &
    \multicolumn{1}{c|}{$g$} &
    \multicolumn{1}{c}{$h$ \emph{added to} $T_h$?}\\
    \midrule
    $4+x$ & $2-x+y+x^2$ & \checkmark\\
    $3x-2y-2x^2$ & $x-2y-x^2-xy-x^3$ & \checkmark \\
    $4y+x^2+3xy+3x^3$ & $4+x$ & - \\
    $x^2+2xy+3x^3$ & $x-2y-x^2-xy-x^3$ & \checkmark \\
    $4xy+4x^3+x^2y+x^4$ & $4+x$ & - \\
    $4x^3+x^4$ & $4+x$ & -\\
    $0$ & - & -\\
    \bottomrule
    \end{tabular}
    \vspace*{5mm}
  \caption{Only \ltrs used in Mora normal form}
  \label{table:infinite-nf-only-lcrs}
\end{table}

If we compare it to Table~\ref{table:infinite-nf} we see one more reduction step
introduced by choosing $x-2y-x^2-xy-x^3$ as reducer in the second reduction
step, since \lcrs are not allowed.
\end{example}

% \todo[inline]{Is it always possible with reduction defined as in
% Definition~\ref{def:reduction} that we first do the lead term reductions and
% then the lead coefficient reductions?}
Note that Lemma~\ref{lem:ltrs-are-enough} shows that \lcrs are, from the
theoretical point of view, not needed. A \lcr corresponds
to a first step in the Euclidean algorithm when calculating $\gcd\left(\lc f, \lc
g\right)$ which will be done in the algorithm when considering $\gpoly fg$.
Still, it has a strong impact on the performance of the algorithm in practice:
Cutting the lead coefficient down as much as possible means that the element
might be used more often for reduction purposes. Moreover, generating new \spts
and \gpts with it leads to lower lead coefficients there. In some sense \lcrs are
the bridge between an \ltr of $f$ by $g$ and the $\gpoly fg$. In one specific
situation we can go directly from one to another, without the need of a bridge. 
Lemma~\ref{lem:ltrs-are-enough} also gives a hint to this situation stated in
the following.
\footnote{This idea is also implemented in the computer algebra system
\macaulay. We have discovered it independently and since we have not
found any proof for the statement we give one here.}
during the reduction process over the integers.
\begin{lemma}
Let $f \in \epr$, $G \subset \epr$ finite and $g\in G$ such that
$\hm g = \hm f$. Then it holds that
\[\left\langle f,g\right\rangle
  = \left\langle \spoly fg, \gpoly fg\right\rangle.\]
  \label{lem:m2-replace-trick}
\end{lemma}

\begin{proof}
Let $u,v, d \in \Z$ such that
\begin{equation}
u\, \lc f + v\, \lc g = d = \gcd\left(\lc f, \lc g\right).
\label{eq:gcd}
\end{equation}
We can then write
\begin{center}
$
\begin{array}[]{rcccccc}
\gpoly fg &= & u& f& + &v& g.\\
\spoly fg &=& \frac{\lc g}{d}& f& - &\frac{\lc f}{d}&g.
\end{array}
$
\end{center}
In order to show the statement we have to proof that $(f,g)$ and $\left(\gpoly fg,
\spoly fg\right)$ generate the same $\Z$-lattice. So, in the above
representation of $\gpoly fg$ and $\spoly fg$ in terms of $f$ and $g$ we have to
show that the corresponding coefficient matrix is invertible, i.e. has
determinante $\pm 1 \in \Z$:
To see this we set
\[M := \begin{pmatrix} u & v\\ \frac{\lc g}{d} & -\frac{\lc
  f}{d}\end{pmatrix}.\]
Finally, we compute
\[\det\left(M\right) = - u\, \frac{\lc f}{d}
- v\, \frac{\lc g}{d} = -\frac 1d \left(u\, \lc f + v\, \lc g\right)
  \stackrel{(\ref{eq:gcd})}{=} -1.\]
\end{proof}

How to use Lemma~\ref{lem:m2-replace-trick} in \bba ? The idea is that whenever
we reduce a new element $f$ we check if there exists a reducer $g\in G$ with
$\hm f = \hm g$, but $\lc g \nmid \lc f$. In this situation we do a $2$-by-$2$
replacement:
\begin{enumerate}
\item Compute $\gpoly fg$ and replace $g\in G$ by $\gpoly fg \in G$. Clearly,
already genrated pairs with $g$ as generator have to adjusted respectively.
\item Compute $\spoly fg$ and replace $f$ by $\spoly fg$. Note that we have not
changed the degree of $f$, but probably only multiplied $f$ with some
coefficient. With the newly defined $f$ we again enter the reduction process and
see, if we can further reducer it.
\end{enumerate}
This has two main advantages to the usual reduction process that would compute
only an \lcr of $f$ w.r.t. $g$:
\begin{enumerate}
\item We directly compute the \gpt of $f$ and $g$
whereas the before mentioned \lcr would represent only one step in the Euclidean
algorithm for reaching $\gpoly fg$. So we can directly replace $g$ with
$\gpoly fg$ which leads to smaller coefficients and multiples during the pair
generation. Furthermore, $\gpoly fg$ reduces other elements at least in all
situations $g$ would reduce, but it can possibly fulfill more \ltrs due to its
smaller lead coefficient.

Furthermore, since $\hm{\gpoly fg} = \hm g$ we can replace all \spts already generated
with $g$, again giving smaller coefficients in upcoming reduction processes.
Even more, using $\gpoly fg$ we are possibly able to render more \spts useless
applying the chain criterion, due to the smaller lead coefficient.
\item For $f$ the advantage is that we are not stuck with a \lcr only, but we
can go on and perform the \ltr $\spoly fg$ and thus directly lower the lead term
without the need of adding $f$ to the basis, which would blowi up pair
generation.
\end{enumerate}

\begin{remark} \
\begin{enumerate}
\item Note that $\hm f = \hm g$ is an essential condition for the correctness of
Lemma~\ref{lem:m2-replace-trick}. If, for example, only $\hm g \mid \hm f$
holds such that $\lambda = \frac{\hm f}{\hm g} > 1 \in \epr$, then we can only
recover $f$ and $\lambda g$ via $\spoly fg$ and $\gpoly fg$, but we are no
longer able to recover $g$.
\item  Overall applying \lcrs has a huge effect on running time: In most examples we
    get a speedup factor of $3$. If we even apply coefficient reductions to the
    tail terms of the newly added element to the basis, we get another factor of
    $3-5$.
% \item Lemma~\ref{lem:m2-replace-tricks} cannot be applied as often as \lcrs, but
% it helps in several ways:
% \begin{enumerate}
% \item We directly get the smallest possible lead coefficient for $f$ and $g$ and
%     not only a slightly reduced one.
% \item We can get rid of a lot of \spats and \gpats since we get as lead
% coefficient the gcd of $\lc f$ and $\lc g$ which helps the chain criterion to
% remove useless data.
% \end{enumerate}
\end{enumerate}
\end{remark}

When we enter the reduction process of an element $f$ in \bba we search for
reducers in the following order:
\begin{enumerate}
\item Is there $g \in G$ for a \ltr of $f$? If so we can cut down the lead
term of $f$ without the need of multiplying $f$ with any coefficient $\neq 1$.
\item Is there are $g \in G$ fulfilling Lemma~\ref{lem:m2-replace-trick}? If so
we can cut down the lead term of some coefficient multiple of $f$ and we can
further replace $g$ by $\gpoly fg$ leading to a better reducer.
\item Is there $g \in G$ for a \lcr of $f$? We cannot cut down the lead term of
$f$, but at least we can reduce the lead coefficient before adding $f$ to $G$
and generating new \spts and \gpts.
\end{enumerate}

% Over fields the concept of a top-reduction is a special case of an \spt
% construction: Instead of multiplying $f$ and $g$ to reach and to cancel out
% $\lcm\left(\hd f, \hd g\right)$ the process is restricted to removing $\hd f$.
% This means that we need $g$ such that $\hd g \mid \hd f$:
% \[\spoly f g =  1 \cdot f - \frac{\hd f}{\hd g} g = f - \frac{\hd f}{\hd g} g.\]
% Over the integers, Definition~\ref{def:reduction} does not completely mimic this
% situation: Even for \spts the concept is not complete: Over fields divisibility
% of the coefficients is a symmetric problem: If $a \in K\setminus\{0\}$ then also
% $\frac{1}{a} \in K\setminus\{0\}$. Over the integers this symmetry is broken,
% but Definition~\ref{def:reduction} does not recover it correctly: We cannot
%   compute a reduction of the form $a f - tg$ where $a \in \Z$, $t \in \epr$ and
% $t \cdot \hm g = \hm f$, $\lc
% g \nmid \lc f$, \emph{but} $\lc f \mid \lc g$ such that $a \cdot \lc f= \lc g$.
% Still, this reduction takes place namely as an \spt $\spoly f g$ that is
% generated once $f$ is added to the intermediate basis. Still, we do not need $f$
% in the basis at all. Even more, Definition~\ref{def:reduction} includes \lcrs: If $\hm g \mid
% \hm f$ but $\lc g \nmid \lc f$ and $\lc f \nmid \lc g$ then we can do a division with rest on the lead
% coefficient of f: Find $a,b \in \Z$ such that $ \lc f = a \lc g +b$ where $a,b
% \neq 0$ and $b \prec \lc f$. Let $h$ be the element resulting from the
% corresponding top-reduction step
% \[h := f - a \frac{\hm f}{\hm g} g.\]
% Then $\hm h = \hm f$ and $\lc h = b$. The problem is that this does not
% correspond to $\gpoly f g$:
% \[h' :=\gpoly f g = c f - d \frac{\hm f}{\hm g} g\]
% such that $\gcd\left(\lc f, \lc g\right) = c\,\lc f - d\,\lc g$ and thus $\hm{h'}
% = \hm f = \hm h$ but $\lc{h'} = \gcd\left(\lc f, \lc g\right)$ which is in
% general not the same as $\lc h$. Take, for example, $\lc f = 44$ and $\lc g = 10$
% then $\lc h = 44-4\cdot 10 = 4$ whereas $\lc{h'} = \gcd\left(44,10\right) = 2$.
% The problem is that we the top-reduction process from
% Definition~\ref{def:reduction} we will not reach $h'$ in this example: Once we
% reached $h$ we would not apply another reduction with $g$: $\hm g \mid \hm h$,
% but $\lc h = 4 = 1 \cdot 10 - 6$ thus $b \succ \lc h$ which is not allowed due to
% Definition~\ref{def:reduction}. But don't we need to reach $h'$ in order to
% compute a strong \stb? Yes, and we would reach it after having added $h$ to the
% intermediate basis and considering the \gpt $\gpoly g h$ later on. This is
% exactly what Lemma~\ref{lem:ltrs-are-enough} tells us. Still,
% this introduces useless overhead, bloating up the basis and the pair set.
%
% Thus we propose new kinds of reduction steps:
%
% \begin{definition}
% Let $f \in \epr$ and let $G \subset \epr$ be a finite subset.
% \begin{enumerate}
% \item If there exists $g\in G$ such that $\hm g \mid \hm f$ and
% either $\lc g \mid \lc f$ or $\lc f \mid \lc g$ then $af - btg$ with
% $(a,b)\in \{1\} \times \Z$ or $(a,b) \in \Z \times \{1\}$ is called a
% \emph{top-\sltr} of $f$ (w.r.t. $g$).
% \item If there exists $g\in G$ such that $\hm g \mid \hm f$, $\lc g \nmid \lc
% f$ and $\lc f \nmid \lc g$ then $af + b \frac{\hm f}{\hm g} g$ with $\gcd\left(\lc f, \lc g\right) =
% a\,\lc f + b\,\lc g$, $a,b \in \Z$, $b \neq 0$ amd $b \prec \lc f$ is a
% \emph{top-\gcdr} of $f$ (w.r.t. $g$).
% \item A top-reduction which allows top-\sltr steps as well as top-\gcdr steps is
% called \emph{strong} or a \sr.
% \end{enumerate}
% \end{definition}
%
% From Lemma~\ref{lem:ltrs-are-enough} we can easily deduce the following
% statement:
%
% \begin{corollary}
% Instead of top-reductions defined by Definition~\ref{def:reduction} we can use
% strong reductions in the normal form algorithms~\ref{alg:nfg} and~\ref{alg:nfl}.
% Applying these normal form algorithms in Algorithm~\ref{alg:bba} we compute a
% strong \stb for a given input ideal w.r.t. the given monomial order.
% \end{corollary}

% \begin{lemma}
% Let $G \subset \epr$ be a finite subset and let $f \in \epr$.
% If there is no \ltr for $f$ w.r.t. $G$ then there is no \ltr for $f'$ w.r.t. $G$
% where $f'$ is the result of finitely many \lcrs w.r.t. $G$ starting with $f$.
% \label{lem:ltr-lcr}
% \end{lemma}
%
% \begin{proof}
% If there does not exist any further \ltr for $f$ w.r.t. $G$ then we have to
% distinguish two cases:
% \begin{enumerate}
% \item If $\nexists g \in G$ such that $\hm g \mid \hm f$ then there also does
% not exist any \lcr of $f$ and the whole top-reduction process of $f$ w.r.t. $G$
% is done.
% \item If $\exists g \in G$ such that $\hm g \mid \hm f$ then $\lc g \nmid \lc f$
% as otherwise $g$ would perform an \ltr with $f$. If $\lc g \nmid \lc f$ then we
% can set $a = \gcd\left(\lc f, \lc g\right)$ and we compute a corresponding
% \end{enumerate}<++>
% \end{proof}<++>
